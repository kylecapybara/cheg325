\documentclass{article}

\input{preamble}
\usepackage{tikz-3dplot}
\usepackage{booktabs}
\usepackage{titling}
\usepackage{fancyhdr}
\usetikzlibrary{patterns}

\usepackage{tcolorbox}
\usepackage[export]{adjustbox}
\usepackage{tikz}

\tcbset{colback=blue!7!white}
\tcbsetforeverylayer{colframe=blue!75!black}


\pretitle{\vspace{-4em}\begin{flushleft}\LARGE} % Adjust the size and alignment
    \posttitle{\end{flushleft}}
    \preauthor{\begin{flushleft}\large} % Adjust the size and alignment
    \postauthor{\hspace{2em} \large \thedate\end{flushleft}} % Place author and date on the same line
    \predate{} % Remove the default date formatting
    \postdate{} % Remove the default date formatting

\pagestyle{fancy}
\fancyhf{} % Clear all header and footer fields
\fancyfoot[R]{\thepage} % Right-align the page number in the footer
\renewcommand{\headrulewidth}{0pt} % Remove the header rule
\renewcommand{\footrulewidth}{0pt} % Remove the footer rule
\setlength{\fboxrule}{1pt}

\geometry{
    top=2cm,    % Top margin
    bottom=3cm, % Bottom margin
    left=2.5cm, % Left margin
    right=2.5cm % Right margin
}
\title{\bfseries CHEG325 Aspen Homework 4 Benzene-Toluene}
\author{Sebastian Blough and Kyle Wodehouse}
\date{}

\begin{document}
\maketitle
\noindent
We obtained the data from Aspen's NIST TDE via adding Benzene and Toluene as components, selecting binary mixture in the TDE, and locating the isothermal VLE.
\begin{figure}[H]
    \centering
    \includegraphics[width=0.3\textwidth, frame]{img/1.JPG}
\end{figure}

\begin{figure}[H]
    \centering
    \includegraphics[width=0.7\textwidth, frame]{img/2.png}
\end{figure}

\begin{figure}[H]
    \centering
    \includegraphics[width=0.7\textwidth, frame]{img/3.png}
\end{figure}

The last piece of information we need is the pure component vapor pressures at this temperature. Luckily, our data set includes when the mole fraction is 0 and 1 so those points are the pure component vapor pressures. Now this data may be looked at inside a jupyter notebook on the next page.


\end{document}